\section{互联网文本信息分析与建模的研究}
本章将从互联网文本信息的特征入手,详细分析与研究各种特征下的文本建模方法。根据第二章的分析,工作主要集中在对文本的主题进行建模与分析,根据不同文本的特征,包括长文本、短文本、文本间相关性以及动态文本等方面,提出有针对性的主题分析方法。利用主题模型对文本进行建模的好处有两个,一是可以对高维度的词汇向量进行降维,也就是用低维度的主题向量来表示文档;二是完全遵循用户生成文档的习惯,一般会先选取几个文档的主题,然后根据主题来生成单词,而主题模型则是很好地模拟了这一过程。

\subsection{长文本信息的分析建模方法}
所谓的长文本即指

\subsection{短文本信息的挖掘与建模方法}

\subsection{关联性文本的挖掘与建模方法}

\subsection{动态更新文本的挖掘与建模方法}

\subsection{小结}
