\section{用户兴趣表示与匹配的研究}
本章将从用户兴趣的特征作为切入点,对每种特征进行深入分析,提出一套完整的兴趣表示方法。根据第二章中的讨论,兴趣具有最明显的两个特征是关联性和动态性。前者说明的是兴趣概念的关系如同名词之间的关系一样具有同义词、上义词和下义词等,文中提出了一种静态兴趣分类树的模型来充分表示这个特征。后者说明的是用户兴趣会随时间变化而变换,并且分为长期兴趣和短期兴趣,所以文中提出了一种动态兴趣伸展树的模型来解决这一变化的过程。此外,为了使系统能够自动地学习用户的兴趣爱好,以便将来可以为用户推荐兴趣(主题)相似的资源,文中提出了一种学习和推荐的方法来完成这个工作。

\subsection{基于文本主题的兴趣点描述方法}
在正式讨论兴趣树的细节之前,先对文本中涉及兴趣相关的概念进行说明和形式化定义。

给定大小为$V$的词汇表$\mathcal{V}=\{v_1,...,v_V\}$,对于大小为$K$的兴趣全集$\mathcal{I}=\{\vec{in}_1,...,\vec{in}_K\}$,其中$\vec{in}_i~(i\in [1,K])$为第$i$个兴趣点。在本文中,兴趣点的名称$name(\vec{in})$是一个字符串,兴趣点的值$value(\vec{in})$看做是从兴趣变量$\vec{x}$抽样出的一个观测值,兴趣变量$\vec{x}$服从Dirichlet分布,即:
\begin{equation*}
  p(\vec{x}|\vec{\alpha})=Dir(\vec{x}|\vec{\alpha})=\frac{\prod_{i=1}^{V}\Gamma(\alpha_i)}{\Gamma(\sum_{i=1}^{V}\alpha_i)}\cdot \sum_{i=1}^{V}x_i^{\alpha_i-1},~~(\vec{\alpha}=(\alpha_1,...,\alpha_V))
\end{equation*}
可以看出,兴趣点$\vec{in}$是一个$V$维的向量,向量中每个值表示当前词汇对兴趣的权重。因此,对于两个兴趣点$\vec{in}_i$和$\vec{in}_j$\$,它们之间的相似度$d_{i,j}$可以用余弦相似度来衡量:
\begin{equation*}
  d_{i,j}=cos(\vec{in}_i,\vec{in}_j)
\end{equation*}

要说明的是,这里的兴趣点的表示方式本质上是继承自文本中的主题,在下文没有特别说明的情况下,可以认为兴趣点等价于主题。同时,关于如何估计兴趣变量$\vec{x}$参数的方法就等同于如何在给定的语料库中训练出主题分布$\Phi$。所以,在这一章中,可以默认提取兴趣点的方式都已经存在,重点是如何利用这些兴趣点构建用户的兴趣树。

\subsection{静态兴趣分类树的表示方法}
这一节主要针对兴趣间的关联性问题,对整个兴趣集建立一个描述模型,从而能够展现出兴趣间的关系。从兴趣点的名称来看,这些名称一般为名词,自然而然地想到了兴趣点之间应该是一种层次关系,于是本节提出了用静态兴趣分类树的方式进行描述整个兴趣集。

\subsubsection{静态兴趣分类树的构建}
在第二章中提出了构建静态兴趣分类树的方法,

\subsubsection{静态兴趣分类树的更新}

\subsubsection{基于静态分类树的兴趣向量匹配}

\subsection{动态兴趣伸展树的表示方法}

\subsection{基于用户兴趣的自动推荐方法}

\subsection{小结}
