\section{结论与展望}

\subsection{总结}
本论文主要对基于P2P网络的信息自主流动机制进行了深入的研究和分析,首次提出了信息自主流动这个机制应该包含的三个层次性模块:信息资源的描述与构建、用户兴趣的表示与匹配以及P2P下的信息自主流动机制。首先,本文从这三个方面出发,调研了国内外的研究现状,并对其中的技术模型、适用范围和优点缺点进行了分析和比较,并提出了在实现信息自主流动机制上会遇到的问题和挑战。通过这些分析之后,本文着手对三个模块进行深入的研究,一方面需要使每一模块各自能够处理互联网上尽可能多的信息,另一方面又要保证三个模块相互之间能够完全兼容,从而符合信息自主流动这一机制的主要目标。

\begin{enumerate}
  \item 信息资源的描述与构建:考虑到互联网资源的类型十分多且差异性很大,本文主要针对文本资源进行分析和构建。同时,高维度和稀疏性是纯文本的两个重要特征,因此隐藏在文本和单词之间的主题特征是一个既可以充分代表文本信息的表示方式,也降低了很多维度,为后续计算提供了方便。因此在第三章中,通过对互联网上四种不同类型的文本资源提出了不同的主题分析方法。针对长文本资源,本文提出了一种基于标签的主题识别算法,这种方法可以对新给定的文本自动分配带有标签的主题。对于短文本资源,本文以新型的弹幕视频为例,利用纯文本弹幕信息对视频的精彩镜头进行主题分析和提取,从而实现了对多媒体资源建立文本索引的可能。对于文本网资源,主要体现在大量互相连接的网页文档,本文提出了一种基于贝叶斯网络的主题识别方法,很好地解决了文本之间相互依赖的关系。最后对于文本流资源,本文提出的方法在一定程度上可以从中发现主题随时间变化的过程。
  \item 用户兴趣的表示与匹配:本文认为用户的兴趣等价于文本的主题,因此对用户兴趣的描述可以直接利用对上文对资源进行主题分析的方法和结果。同时,考虑到用户兴趣具有层次关联和动态变化两个重要特征,本文分别提出了两种兴趣树作为解决方案。其中,静态兴趣分类树(SICT)的作用在于将兴趣和兴趣之间的层级关系描述得十分清晰,文中提出了一种基于SICT的用户兴趣相似度匹配的方法,弥补了普通相似度算法在表达层次关系上的不足。动态兴趣伸展树(DIST)的作用在于描述和存储用户兴趣随时间的变化状态的特征,文中提出了一种基于DIST的用户兴趣相似度匹配方法,弥补了普通相似度算法在表达长期兴趣和短期兴趣的不足。最后,基于上述两种兴趣树,本文正式提出了一种用于描述用户的兴趣的模型,即用兴趣集来代替普通的兴趣向量表示方法。
  \item 基于P2P网络的信息自主流动:根据上文提出的用户兴趣模型,本文随后给出了基于P2P网络的信息自主流动机制,即通过在普通的用户关系层上添加兴趣覆盖网络,通过判断用户兴趣与兴趣、用户兴趣与资源间的相似度,使得源节点能够根据用户兴趣来定位目标节点所需要的资源。
\end{enumerate}

根据上述提出的方法,本文首先对部分理论模型进行了模拟实验和真实实验的验证。在文本主题分析方面,系统可以自动对文本进行识别带有标签的主题,相比普通的主题模型,该方法可以以87.3\%以上的正确率为文本标记至少一个主题,这在实际应用中已经达到了基本要求。在短文本主题分析中,基于弹幕的视频精彩镜头的提取方法在覆盖率和准确率等性能指标上比其他基于相似度和频率的提取方式高出了最多11.2\%的。在用户兴趣模型方面,基于SICT和DIST的兴趣相似度匹配算法相比普通基于余弦相似度的匹配算法在准确率上有了10\%左右的提升。在兴趣覆盖网络方面,本文提出的基于用户兴趣集的资源传播方法相比基于普通好友关系的传播方法有很明显的准确率提升,在减少了不相关信息传播的同时也降低了网络的带宽的开销。

此外,由于信息自主流动机制这个概念是首次在本文中提出,其在实际应用的效果具有很多未知因素,因此本文在最后详细说明了一套基于此机制的原型系统在实现方面的细节和要注意的事项。

总而言之,本文以信息自主流动机制为核心,从三个方面分别展开,详细论述了实现该机制的模型和方法。在研究过程中,涉及了数据挖掘、机器学习、对等计算、形式化理论以及众多工程应用中的编程技术,工作量巨大。最后,通过实验验证了各个环节中提出的方法在评价指标上表现得很好,相比其他普通的方法有了一定的进步。

\subsection{展望}
本文虽然在信息自主流动机制上取得了一定的研究成果,但是仍然有一些可以提升的方面,总结如下:
\begin{itemize}
  \item 带有标签的层次主题分析:如何能够自动化构建SICT是一个值得深入研究的问题,因为互联网中的包含的主题和兴趣十分多,全部靠人工来建立往往是不切实际的。因此,如果对文本主题分析的时候,能够在保证主题与标签一一对应的关系同时,还可以分析出标签之间的关系,就会给SICT的建立带来很大的帮助。
  \item 用户兴趣预测:本文中的兴趣模型仅限于对用户当前兴趣的计算,而没有对未来可能产生的兴趣做一个预测。一种可行的方式是将用户的兴趣视作状态,而用户对资源的操作视为动作,并将用户的反馈作为奖励机制,通过马尔科夫决策过程等强化学习方法自动训练出一个智能agent。
  \item 半中心化兴趣覆盖网络:类似于CDN的架构模式,可以将边缘服务器用作对兴趣的管理,通过对用户兴趣的分类,每个边缘服务器分别负责不同兴趣节点的索引,从而使得每个用户节点本身在存储更少数据的的同时,加快了资源传播的效率。
  \item 信息自主流动机制的实际应用:由于时间有限,本文只是初步实现了具有基本功能的节点系统。因此,对系统方面仍有很大的改进空间,一种可行的方法是将现有的原型投入到一定区域内进行实测,根据用户的反馈再改进该机制的细节,从而实现真正的信息自主流动。
\end{itemize}
