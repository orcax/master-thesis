\section{已经正常的功能}
\subsection{List}
	\begin{itemize}
		\item 中英文封面
		\item 中英文摘要
		\item 目录
	\end{itemize}
\subsection{字体和字号}

	中文默认字体宋体。 \textit{斜体} \textbf{粗体} \\
	Default English font: Time New Roman\\

	 改成别的字体:
	 \begin{itemize}
	 \item {\song \textbackslash song 宋体}
	 \item {\fangsong \textbackslash fangsong 仿宋}
	 \item {\li \textbackslash li 隶书}
	 \item {\hei \textbackslash hei 黑体}
	 \item {\arial \textbackslash arial Arial}
	 \item {\timesnew \textbackslash timesnew Times New Roman}
	 \end{itemize}

	 \vspace*{1em}改变字号:
	 \begin{itemize}
		 \item {\chuhao \textbackslash chuhao 初号 42pt}
		 \item {\xiaochu \textbackslash xiaochu 小初号 36pt}
		 \item {\yihao \textbackslash yihao 一号 28pt}
		 \item {\erhao  \textbackslash erhao 二号 21pt}
		 \item {\xiaoer \textbackslash xiaoer 小二号 18pt}
		 \item {\sanhao \textbackslash sanhao 三号 15.75pt}
		 \item {\sihao \textbackslash sihao 四号 14pt}
		 \item {\xiaosi \textbackslash xiaosi 小四号 12pt}
		 \item {\wuhao \textbackslash wuhao 五号 10.5pt}
		 \item {\xiaowu \textbackslash xiaowu 小五号 9pt}
		 \item {\liuhao \textbackslash liuhao 六号 7.875pt}
		 
	 \end{itemize}
	 
	\subsection{页眉和页脚}
	页眉从中文摘要开始,单数页页眉在右上角,双数页页眉在左上角。
	“摘要”至篇尾,除“正文” 部分外,页眉均为“同济大学 
	硕/博士学位论文 XXX”,如“同济大学 硕/博士学位论文 摘要”;“同 济大学
	 硕/博士学位论文 参考文献”等。正文的单数页页眉为章标题:“第 X 章
	  XXXXXX”,双数页 页眉为论文标题;“同济大学 硕/博士学位论文 中文题目”。
页码从第 1 章(引言)开始按阿拉伯数字(1,2,3......)连续编排,
之前的部分(中文摘要、 Abstract、目录等)用大写罗马数字(I,II,III......)单独编排。



\subsection{图表和表达式}
	表达式主要指数字表达式,也包括文字表达式。表达式需另行起排,并用阿拉伯
	数字分章编号。 序号加圆括号,右顶格排。例如:第 3 章第 2 个表达式(3.2)
	
\subsubsection{图}
	\begin{figure}[h!]
		\centering
		\includegraphics[width=0.5\linewidth]{tongji-logo.png}
		\caption{同济Logo}
		\label{fig:tongji}
	\end{figure}

\subsubsection{表}
		\begin{table}[h!]
		\centering
			\caption{A Table}
			\begin{tabular}{c|c}
			a & b \\ \hline
			c & d \\ \hline
			\end{tabular}
		\end{table}
		
		\begin{table}[h!]
			\centering
				\caption{Tableeeeeeeeee}
				\label{tb:test_tb}
				\begin{tabular}{c|c}
				a & b \\ \hline
				c & d \\ \hline
				\end{tabular}
			\end{table}
	
			\begin{tabular}{p{5cm}}
			hello \\ 
			hello world hello world hello world hello world \\ \hline
			\end{tabular}
	
\subsubsection{公式}
	\begin{equation}\label{eq:test}
	\sum_{i=1}^n a_i \times b_i 
	\end{equation}
	
	\begin{equation}
		\sum_{i=1}^n a_i \times b_i 
		\end{equation}
		
	多行公式:
	\begin{eqnarray}
			\sum_{i=1}^n a_i \times b_i  \nonumber \\
				\sum_{i=1}^n a_i \times b_i 
	\end{eqnarray}
\subsection{交叉引用}
	如公式~\ref{eq:test}所述,这是一个公式。表~\ref{tb:test_tb}是
	一个不长的表格
	
	如图~\ref{fig:tongji}所示,这是一个同济大学的标识。
	
\subsection{其他码农功能}
\subsubsection{算法}
算法流程用的\textit{algorithm2e}包
\begin{algorithm}
	\KwData{this text}
	 \KwResult{如何写算法 \LaTeX2e }
	 initialization\;
	 \While{not at end of this document}{
	  read current\;
	  \eIf{understand}{
	   go to next section\;
	   current section becomes this one\;
	   hello world\;
	   	 \KwContinue
	   	 \KwBreak 
	   }{
	   go back to the beginning of current section\;
	  }
	 }
	 \caption{如何写算法}
\end{algorithm}

\subsubsection{代码}
use package \textit{listings}. Customized format can be set
%\lstset{breaklines=true,tabsize=4,extendedchars=false,escapechar=\%}
\lstset{basicstyle=\ttfamily\small ,frame=single,captionpos=t}
%\lstinputlisting[language=Python]{src.py}
\begin{lstlisting}[language=Python, caption=一段python代码] 
import os
import sys
import re
import string

name = sys.argv[1]
data = open(name)
data2 = file('2'+name,'wr')
d1 = list()
d2 = list()
d3 = list()

for each_line in data:
	each_line=each_line.replace('\n','')
	elems = each_line.split('\t')
	d1.append(elems[0])
	d2.append(elems[1])
	d3.append(elems[2])
\end{lstlisting}



\section{章节标题}
	章节标题 \textbackslash section \\
	show一下段落层次
\subsection{一级标题} 
	一级标题 \textbackslash subsection
	
\subsubsection{二级标题}
	二级标题 \textbackslash subsubsection
	
\paragraph{三级标题}
	三级标题 \textbackslash paragraph
看看三级标题有没有断行 ----- 结论 断行啦\\




\section{基本功能测试}
\subsection{待解决问题}
	``这个貌似不对了''
		

	
\subsection{引用}
	随便找一篇文章来测试。中文\cite{余胜泉2000}。 英文\cite{liu2010system}。
	bib生成的引用中英文有点凌乱。。。治标不治本的方法是把bbl贴进来改一改.
	
	暂时用了IEEEtran的bibliography style。


\subsection{附录}

	图和表的目录。。。。。。\\
	某一个表达式:
	\begin{equation}
	 \sum_{i=1}^{n}x_i = 10
	\end{equation}
	




\section{图表公式再测一次}

\subsection{要个subsection吧}

\subsubsection{图}
	\begin{figure}[h!]
		\centering
		\includegraphics[width=0.5\linewidth]{tongji-logo.png}
		\caption{同济Logo}
		\label{fig:tongji2}
	\end{figure}

\subsubsection{表}
		\begin{table}[h!]
		\centering
			\caption{A Table}
			\begin{tabular}{c|c}
			a & b \\ \hline
			c & d \\ \hline
			\end{tabular}
		\end{table}
		
		\begin{table}[h!]
			\centering
				\caption{Tableeeeeeeeee}
				\label{tb:test_tb2}
				\begin{tabular}{c|c}
				a & b \\ \hline
				c & d \\ \hline
				\end{tabular}
			\end{table}
			
\subsubsection{公式}
	\begin{equation}\label{eq:test2}
	\sum_{i=1}^n a_i \times b_i 
	\end{equation}
	
	\begin{equation}
		\sum_{i=1}^n a_i \times b_i 
		\end{equation}
		
	多行公式:
	\begin{eqnarray}
			\sum_{i=1}^n a_i \times b_i  \nonumber \\
				\sum_{i=1}^n a_i \times b_i 
	\end{eqnarray}
