\section{基于P2P网络的信息自主流动机制的方案概述}
本章将阐述基于P2P网络的信息自主流动机制的基本方案。在整套机制中,需要涉及三大研究模块:在文本信息分析与建模的研究中,主要针对互联网中纯文本的信息的非结构化问题,通过数据挖掘和机器学习等方法从中抽取出一部分能够充分表示原文本的特征,并且提出一套适用于各种类型文本的描述方式;在用户兴趣表示与匹配的研究中,主要结合文本信息与用户反馈提出一种具有代表性的兴趣构建方法,从而可以进行用户与用户、用户与资源之间的兴趣相似度计算。在P2P网络路由与发现的研究中,主要利用上述提出的针对资源和用户兴趣的模型,以现在网络为底层架构,构建一层用户兴趣覆盖网络,并提出了网络构建、动态更新与搜索发现等基本方法。最后,基于上述三个模块,本文还设计并实现了一个简易实现的原型系统,主要包括架构设计和功能设计,以此来说明信息自主流动机制的可行性和实际应用价值。

\subsection{文本信息分析与建模}
按照互联网资源的类型可以分为:文本、图片、音乐、视频等。虽然如今涌现了越来越多的诸如音频和视频的流媒体资源,但是互联网上的资源大多任然以文本形式存在,或者可以用文本来代替原有的非文本资源。不仅限于传统的新闻、博客是文本信息,连图片也可以用很成熟的技术来转换成文本,比如百度识图\footnote{http://stu.baidu.com/}就可以自动识别出图片中的物体,即使是视频资源也可以用元信息、标题、评论和弹幕等文本信息进行概括。因此下文主要针对互联网文本信息进行深入研究,并假设其他类型的资源均可以通过已有成熟的方法转换成文本信息。这一小节将根据互联网上四种不同类型的文本,分别提出文本挖掘与建模的方法。

普通的新闻、博客等文章都归类为长文本信息,这类资源一般都是纯文本数据,因此具有两个明显的特征:稀疏性和高维度。举个例子来说:假设给定一由$|\mathcal{V}|$个不同词汇组成的字典$\mathcal{V}$,在由$M$篇文章组成的语料库中,每篇文章$d$的用词都属于词典$\mathcal{V}$,且每篇文章的单词数量不少于$n (0<n\ll W)$。如果直接简单地将文档表示成关于词汇的向量,向量中的每个值表示该词汇在文中的词频,如果该词汇没有在文中出现,则向量中对应的值为0。那么有两点是显而易见的:a)文档向量的维度为$|\mathcal{V}|$,在正常情况下,$|\mathcal{V}|$可以达到百万数量级($10^6$);b)文档向量中最多只有$n$个值大于0,一般情况下,$n$只有几百数量级($10^2$)。这就是纯文本的高维度和稀疏性带来的问题。为了解决这个问题,本文采用基于概率图的主题模型来进行分析,这里先形式化地定义几个重要的基本概念。

对于包含$M$篇文档$d$的语料库$\mathcal{D}$,有$\mathcal{D}=\{d_1,d_2,...,d_M\}$。词汇表$\mathcal{V}$中包含了$W$个不同的词汇$v$。对于文档$d_i$,其中每个单词$w$都取自于$\mathcal{V}$,并且单词可以重复,即$w_i=w_j=v~~(i\ne j, v\in \mathcal{V})$。那么,对于长度为$N$的文档$d$表示为关于单词的向量,$d=\vec{w}=(w_1,w_2,...,w_N)$。

为了简化模型,我们认为文档单词之间是没有先后关系的,换句话说,每个单词相互独立,一个单词出现的概率不会收到另一个单词的影响。因此,我们可以计算文档$d$的生成概率$p(d)$:
\begin{equation}
  p(d)=p(\vec{w})=p(w_1,w_2,...,w_N)=p(w_1)p(w_2)...p(w_N)
\end{equation}
而对于语料库$\mathcal{D}$的生成概率$p(\mathcal{D})$为:
\begin{equation}
  p(\mathcal{D})=p(d_1)p(d_2)...p(d_M)=p(\vec{w_1},\vec{w_2},...,\vec{w_M})
\end{equation}

如果用词频来描述文档的话,$d_m=\vec{n_m}=(n_{m,1},n_{m,2},...,n_{m,W})$,其中$n_{m,i}~~(1\le i\le W)$表示文档$d_m$中出现词汇$v_i$的频数。于是,整个语料库$\mathcal{D}$可以表示为一个$M*W$维的矩阵$X$,即:
\begin{equation*}
  X=\left(
  \begin{array}{cccc}
    n_{1,1} & n_{1,2} & \cdots & n_{1,W} \\
    n_{2,1} & n_{2,2} & \cdots & n_{2,W} \\
    \vdots & \vdots & \ddots & \vdots \\ 
    n_{M,1} & n_{M,2} & \cdots & n_{M,W} \\
  \end{array}
  \right)
\end{equation*}
其中,$n_{m,v}~~(1\le m\le M,1\le v\le W)$表示文档$d$中拥有词汇$v$的数量。

假设词汇$v_i$出现的先验概率为$p_i$,词汇表$\mathcal{V}$中所有词汇组成的先验概率为$\vec{p}=(p_1,p_2,...,p_W)$。又文档$d=\vec{n}=(n_1,n_2,...,n_W)$,那么$d$生成的概率为:
\begin{equation}
  p(d)=p(v_1)^{n_1}p(v_2)^{n_2}...p(v_W)^{n_W}=\prod_{i=1}^{W}p_i^{n_i}
\end{equation}

从整个语料库来看,假设每个词汇$v_i$出现的次数为$n_i$,那么语料库$\mathcal{D}$又可以表示为$\vec{n}=(n_1,n_2,...,n_W)$,在语料库生成过程中,我们可以把$\vec{n}$看做一个是服从多项分布的随机变量,而$(n_1,n_2,...,n_W)$是一组具体的观测值,得到如下:
\begin{equation}
  \label{eq:vocmulti}
  p(\vec{n})=mult(\vec{n}|\vec{p},N)=
  \begin{pmatrix}
    N \\
    \vec{n}
  \end{pmatrix}
  \prod_{i=1}^Wp_i^{n_i}
\end{equation}

现在的问题是如何利用语料库$\mathcal{D}$已知的数据来估计未知的参数$\vec{p}$?$\vec{p}$的实际意义是词汇表$\mathcal{V}$中每个词汇各自出现的频率。

一种简单的方法是利用最大似然估计(MLE)来估计参数$p_i$的值得到:
\begin{equation}
  \widehat{p_i}=\frac{n_i}{N}
\end{equation}
其中N为语料库的单词总数。这个方法的假设前提是$\vec{p}$的值本身是一个实数常量,换句话说,任何其他语料库$\mathcal{D}'$的生成过程也是依赖于同一个词汇表来生成的。

但是,这个假设对于互联网上的文本是不正确的,有以下三点主要原因:
\begin{itemize}
\item 多类型文档:互联网上的文档类型包括了新闻、技术博客、心情随笔等等,这些文档的用词肯定是很不一样的。比如,对于新闻类型的文档,很少能使用主观色彩的词汇,因此如果将所有新闻文档组成一个词汇表的话,主观性形容词的先验概率是十分低的。但是,这与心情随笔类型的文档恰恰相反,因为此类文档大多是抒发作者内心的心情的,如果将此类型文档组成一个词汇表的话,主观性形容词的先验概率十分高。
\item 多元化主题:即使是同一类型的文档也有各种不同主题,比如新闻中就包含体育、娱乐、经济、社会、科技等各种主题。对于不同的主题,生成文档的词汇表应该也是不同的。如果将上述$\vec{p}$中的值认为是常数的话,相当于所有文档都是从同一个主题中生成的,这显然违背互联网资源的特征。
\item 个性化写作:即使上述两点都保持一致,每个作者写作风格和兴趣爱好同样会影响文档生成的过程。比如说,将每个作者写的所有技术博客各自组成一个词汇表,由于用词习惯等因素,那么这些词汇表中的词汇出现概率肯定也千差万别。
\end{itemize}

总而言之,一篇文档的生成过程受到包括文档类型的制约、主题风格的用词和用户个性化的定制等几方面的影响。因此,基于这些原因,对互联网文本的表示不能简单地用一个词频向量$d=\vec{n}=(n_1,n_2,...,n_W)$来表示。

于是,我们考虑到了两个重要的因素。第一,存在不止一个的词汇表,那么$\vec{p}$就应该被认为是一个随机变量,所以语料库生成的概率应该是受词汇表影响的一个条件概率$p(\mathcal{D}|\vec{p})$。为了得到随机变量$\vec{p}$的分布和参数,我们可以根据贝叶斯定理得到:
\begin{equation}
  p(\vec{p}|\mathcal{D})=\frac{p(\vec{p})p(\mathcal{D}|\vec{p})}{p(\mathcal{D})}
\end{equation}
其中,$p(\mathcal{D}|\vec{p})$就是之前的$\vec{n}=(n_1,n_2,...,n_W)$,即语料库中每个词汇出现的频数,且和公式\ref{eq:vocmulti}一样,是一个服从多项分布的随机变量。为了计算方便,这里使得$p(\vec{p}|\vec{\alpha})$成为多项分布的共轭先验,也就是Dirichlet分布,得到:
\begin{equation}
  p(\vec{p}|\vec{\alpha})=Dir(\vec{p}|\vec{a})=\frac{1}{\Delta(\vec{\alpha})}\prod_{i=1}^{W}p_i^{\alpha_i-1},~~\vec{\alpha}=(\alpha_1,...,\alpha_W)
\end{equation}
式中的$\Delta(\vec{\alpha})$是事先给定的归一化因子:
\begin{equation*}
  \Delta(\vec{\alpha})=\int \prod_{i=1}^{W}p_{i}^{\alpha_i-1}d\vec{p}
\end{equation*}
至此,对于给定$\vec{\alpha}$,语料库$D$的生成概率为:
\begin{eqnarray}
  p(\mathcal{D}|\vec{\alpha})&=&\int p(\mathcal{D}|\vec{p})p(\vec{p}|\vec{\alpha})d\vec{p} \\
                             &=&\int (\prod_{i=1}^{W}p_{i}^{n_i}\frac{1}{\Delta(\vec{\alpha})})(\prod_{i=1}^{W}p_{i}^{\alpha_i-1})d\vec{p} \\
                             &=&\frac{\Delta(\vec{n}+\vec{\alpha})}{\Delta(\vec{\alpha})}
\end{eqnarray}

第二,除了词汇表是随机取一个的之外,我们还可以为每个文档加上主题,这里的主题可以近似看做不同的词汇表,或者更加确切应该是词汇表中词汇的分布。对于互联网文本来说,每篇文档的主题应该不只一个,而且多个主题之间的比重也是不一样的。因此,文档的生成过程应该是先给定一个主题的分布,表示每个主题在文档中占有的比重,然后再根据这个主题的分布随机选取一个主题(也就是词汇表),最后根据这个词汇表生成一个单词。在生成第二个单词的时候,再根据主题分布选择一个主题,并用相应的词汇表生成一个单词。以此类推直至完成一篇文章的生成过程。这个生成过程与LDA背后的思想类似,因此就直接引用LDA模型的算法进行求解。

\begin{figure}
\centering
\includegraphics[width=0.8\textwidth]{lda.png}
\caption{LDA主题模型的概率图表示方式}
\label{fig:lda}
\end{figure}

如图\ref{fig:lda}所示,对于$M$篇文档、$K$个主题的语料库,文档-主题的分布为$\vec{\theta_m}~~(1\le m\le M)$,主题-词汇的分布为$\vec{\phi_k}~~(1\le k\le K)$。这两个随机变量与上文的词汇表类似,且都服从Dirichlet分布,即:
\begin{eqnarray}
  p(\vec{\theta}|\vec{\alpha})=Dir(\vec{\theta}|\vec{\alpha})=\frac{1}{\Delta(\vec{\alpha})}\prod_{i=1}^{K}\theta_{i}^{\alpha_i-1} \\
  p(\vec{\phi}|\vec{\beta})=Dir(\vec{\phi}|\vec{\beta})=\frac{1}{\Delta(\vec{\beta})}\prod_{i=1}^{W}\theta_{i}^{\beta_i-1}
\end{eqnarray}

在文档$d$中生成第$n$个单词的时候,首先要从文档-主题分布$\vec{\theta}$中抽样得到的主题编号为$z_{d,n}$,然后选取编号为$z_{d,n}$的主题-词汇分布$\vec{\phi}$,并抽样得到的一个单词$w_{d,n}$。由上面的理论可知,从Dirichlet分布抽样可以得到的观测值服从多项分布,记$\vec{x_d}=(x_{d,1},...,x_{d,K})$,$x_{d,k}~~(1\le k\le K)$表示在文档$d$中,抽样得到的编号为$k$的主题数量;同理,$\vec{y_d}=(y_{d,1},...,y_{d,W})$,$y_{d,v}~~(1\le v\le W)$表示在文档$d$中,抽样得到的词汇$v$的数量。$\vec{y_d}$和$\vec{y_d}$均为多项分布,即:
\begin{eqnarray}
  &&p(\vec{x_d}|\vec{\theta_d})=mult(\vec{x_d}|\vec{\theta_d})=
  \begin{pmatrix}
    N \\ \vec{x_d}
  \end{pmatrix}
  \prod_{i=1}^{K}\theta_i^{x_{d,i}} \\
  &&p(\vec{y_d}|\vec{x_d},\vec{\phi_d})=mult(\vec{y_d}|\vec{x_d},\vec{\phi_d})=
  \begin{pmatrix}
    N \\ \vec{y_d}
  \end{pmatrix}
  \prod_{i=1}^{W}\phi_i^{y_{d,i}}
\end{eqnarray}

有了以上基础之后,我们最终可以得到文档$d_m=\vec{w_m}$的生成概率为\cite{heinrich2005parameter}:
\begin{equation}
  p(d_m|\vec{\alpha},\vec{\beta})=\iint p(\vec{\theta_m}|\vec{\alpha})p(\Phi|\vec{\beta})\cdot \prod_{n=1}^{N_m}p(w_{m,n}|\vec{\theta_m},\Phi)d\Phi d\vec{\theta_m}
\end{equation}
所以整个语料库生成的概率为:
\begin{equation}
  p(\mathcal{D}|\vec{\alpha},\vec{\beta})=\prod_{m=1}^{M}p(d_m|\vec{\alpha},\vec{\beta})
\end{equation}

现在,我们的问题转换成估计参数$\Theta=\{\vec{\theta_1},...,\vec{\theta_M}\}$和$\Phi=\{\vec{\phi_1},...,\vec{\phi_K}\}$。考虑到该问题涉及过多纯统计学知识,下文我们将直接使用成熟的collapsed Gibbs Sampling算法\cite{griffiths2004finding}进行求解,具体公式将不再列出。

\subsection{用户兴趣表示与匹配}
随着互联网上新型的网站越来越多,用户也会被各种各样的信息资源所吸引,而这些资源的主题可以在一定程度上反映出用户的兴趣所在。因此,我们将利用之前提出的文本信息分析与匹配方法先挖掘出隐藏在文本背后的主题,然后,通过发现并结合互联网用户兴趣的特点,提出一种可以全面且有效的表示方法和其对应的匹配机制。

首先,目前最常用的表示用户兴趣的方式是VSM,假设一共存在$K$种兴趣,并且兴趣之间没有关联性,而用户的兴趣则表示为一个$K$维向量$\vec{in}=(w_1,w_2,...,w_K),~~w_i\in [0,1](1\le i\le K)$,$w_i$表示用户对第$i$个兴趣的喜好程度。用VSM来表示用户兴趣的好处是在计算兴趣相似度的时候速度十分快,一般使用余弦相似度就能计算得出两个向量之间的距离。当余弦值越小就表示用户之间拥有更为相似的兴趣爱好。虽然这个方法在大部分情况下都十分快捷有效,但是,考虑到互联网用户的特点,上述方法存在下面两个问题,因此不能用于完全表示用户兴趣。
\begin{itemize}
  \item 兴趣关联性:由于用户的兴趣对应于文本资源中的主题,而主题之间是由相关性的,这种相关性体现在两个方面。一是主题之间的同义或反义,比如,CBA和NBA是两个主题,但是它们之间的共同点是都是篮球比赛,区别在于举办地点不同、规则不同等等。二是主题之间有层级关系,比如篮球和NBA是两个主题,篮球是一个更加广阔的概念,包括了篮球比赛、篮球运动员、实体篮球等,而NBA则是一个相对狭窄的概念,所以NBA这个主题是属于篮球的。
  \item 兴趣动态性:用户的兴趣一般可以分为长期兴趣和短期兴趣。长期兴趣相对稳定,变化较少,构建模型时比较容易。但是短期兴趣是一个难点,因为它会随着时间的变化而变化,而且每次持续的时间也都未知,可以近似把兴趣与时间的关系看成一个随机过程。
\end{itemize}



\subsection{P2P网络构建与发现}

\subsection{基于信息自主流动机制的原型系统}

\subsection{小结}
